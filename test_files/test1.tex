\documentclass{article}

\begin{document}
In many applications of calculus, 
the quadratic formula is used to find 
the roots of a quadratic equation 
$ ax^2 + bx + c = 0 $. 
The solutions to this equation can be found 
using the quadratic formula 
\[ x = \frac{-b \pm \sqrt{b^2 - 4ac}}{2a}.\] 
This formula is derived by completing 
the square and provides a straightforward method to 
find the values of $ x $ that satisfy the equation. 
The discriminant, $ \sqrt{b^2 - 4ac} $, 
determines the nature of the roots: 
if it is positive, 
the equation has two distinct real roots; 
if it is zero, the equation has exactly one real root; 
and if it is negative, the equation has two complex roots.

Understanding the behavior of functions often 
requires analyzing their derivatives. 
For a sequence $\{x_n\}$, 
the next term can be represented as $x_{n+1}$, indicating the dependence on the previous term. 
Similarly, for a function $f(x)$, 
the first derivative $f'(x)$ represents 
the rate of change of 
the function with respect to $x$. 
When dealing with polynomial functions, 
such as $f(x) = ax^n$, the power rule for 
differentiation states that $f'(x) = n \cdot ax^{n-1}$. 
This rule is fundamental in calculus and provides a straightforward method to find the slope of 
the tangent line at any point on the function. 
Additionally, higher-order derivatives, 
like $f''(x)$ and $f^{(3)}(x)$, can be found similarly, 
providing deeper insights into 
the function's curvature and concavity.

A well-known formula in calculus is 
the Taylor series expansion, 
which expresses a function as 
an infinite sum of terms calculated from 
the values of its derivatives at a single point. 
For a function $f(x)$ centered at $x = a$, 
the Taylor series is given by
\begin{equation}
f(x) = f(a) + f'(a)(x - a) + \frac{f''(a)}{2!}(x - a)^2 + \frac{f^{(3)}(a)}{3!}(x - a)^3 + \cdots
\end{equation}
This formula allows us to approximate 
complex functions using polynomials, 
which are easier to work with 
analytically and computationally. 
The accuracy of the approximation increases with 
the number of terms used, 
making the Taylor series a powerful tool in 
both theoretical and applied mathematics.

For certain types of differential equations, 
solutions can be represented in 
terms of special functions. 
One such example is the Bessel function of 
the first kind, 
which satisfies Bessel's differential equation:
\begin{equation}
x^2 y'' + x y' + (x^2 - n^2) y = 0
\end{equation}
where $ y = J_n(x) $ is the Bessel function and 
$ n $ is a constant. 
These functions frequently appear in 
problems with cylindrical or 
spherical symmetry in physics and engineering, 
demonstrating the broad applicability of 
advanced mathematical techniques.
\end{document}